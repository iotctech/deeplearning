\PassOptionsToPackage{unicode=true}{hyperref} % options for packages loaded elsewhere
\PassOptionsToPackage{hyphens}{url}
%
\documentclass[12pt,a4paper,UTF8,twoside]{book}

\usepackage{lmodern}
\usepackage{setspace}
\setstretch{1.25}
\usepackage{amssymb,amsmath}
\usepackage{ifxetex,ifluatex}
\usepackage{fixltx2e} % provides \textsubscript
\ifnum 0\ifxetex 1\fi\ifluatex 1\fi=0 % if pdftex
  \usepackage[T1]{fontenc}
  \usepackage[utf8]{inputenc}
\else % if luatex or xelatex
  \ifxetex
    \usepackage{xltxtra,xunicode}
  \else
    \usepackage{fontspec}
  \fi
  \defaultfontfeatures{Mapping=tex-text,Scale=MatchLowercase}
  \newcommand{\euro}{€}

    \setmainfont[]{Times New Roman}
    \setsansfont[]{Arial}
    \setmonofont[Mapping=tex-ansi]{Inconsolata}
\fi

% use upquote if available, for straight quotes in verbatim environments
\IfFileExists{upquote.sty}{\usepackage{upquote}}{}
% use microtype if available
\IfFileExists{microtype.sty}{%
\usepackage[]{microtype}
\UseMicrotypeSet[protrusion]{basicmath} % disable protrusion for tt fonts
}{}
\usepackage{hyperref}
\hypersetup{
            pdftitle={深度学习文档集},
            pdfauthor={贝塔},
            pdfproducer={Pandoc, R Markdown, TinyTeX, knitr, bookdown, Stan},
            pdfborder={0 0 0},
            breaklinks=true}
\urlstyle{same}  % don't use monospace font for urls
\usepackage[margin=1.18in]{geometry}
\usepackage{longtable,booktabs}
% Fix footnotes in tables (requires footnote package)
\IfFileExists{footnote.sty}{\usepackage{footnote}\makesavenoteenv{longtable}}{}
\usepackage{graphicx}
\makeatletter
\def\maxwidth{\ifdim\Gin@nat@width>\linewidth\linewidth\else\Gin@nat@width\fi}
\def\maxheight{\ifdim\Gin@nat@height>\textheight\textheight\else\Gin@nat@height\fi}
\makeatother
% Scale images if necessary, so that they will not overflow the page
% margins by default, and it is still possible to overwrite the defaults
% using explicit options in \includegraphics[width, height, ...]{}
\setkeys{Gin}{width=\maxwidth,height=\maxheight,keepaspectratio}
\setlength{\emergencystretch}{3em}  % prevent overfull lines
\providecommand{\tightlist}{%
  \setlength{\itemsep}{0pt}\setlength{\parskip}{0pt}}
\setcounter{secnumdepth}{5}
% Redefines (sub)paragraphs to behave more like sections
\ifx\paragraph\undefined\else
\let\oldparagraph\paragraph
\renewcommand{\paragraph}[1]{\oldparagraph{#1}\mbox{}}
\fi
\ifx\subparagraph\undefined\else
\let\oldsubparagraph\subparagraph
\renewcommand{\subparagraph}[1]{\oldsubparagraph{#1}\mbox{}}
\fi

% set default figure placement to htbp
\makeatletter
\def\fps@figure{htbp}
\makeatother

\usepackage[UTF8, heading, fontset=none]{ctex}
\usepackage{amssymb,amsmath,amsfonts,mathrsfs}
% \setCJKmainfont[ItalicFont={KaiTi_GB2312}, BoldFont={SimHei}]{SimSun}
\setCJKmainfont[ItalicFont={CESI_KT_GB2312}, BoldFont={SimHei}]{SimSun}
\setCJKsansfont{SimHei}
% \setCJKmonofont{FangSong_GB2312}
\setCJKmonofont{CESI_FS_GB2312}

\setCJKfamilyfont{heiti}{SimHei}             
\newcommand{\heiti}{\CJKfamily{heiti}}

% \setCJKfamilyfont{kaishu}{KaiTi_GB2312}             
\setCJKfamilyfont{kaishu}{CESI_KT_GB2312}             
\newcommand{\kaishu}{\CJKfamily{kaishu}}

\setCJKfamilyfont{songti}{SimSun}             
\newcommand{\songti}{\CJKfamily{songti}}

% \setCJKfamilyfont{fangsong}{FangSong_GB2312}             
\setCJKfamilyfont{fangsong}{CESI_FS_GB2312}             
\newcommand{\fangsong}{\CJKfamily{fangsong}}

\usepackage{color}
\ctexset{
  chapter/name = {,},
  chapter/number = \arabic{chapter},
  chapter/numberformat = \sf,
  chapter/beforeskip = 12pt,
  chapter/afterskip = 18pt,
  chapter/fixskip = true,
  chapter/format += \sf\zihao{3},
  section/numberformat = \rm,
  section/format += \sf\zihao{4}\raggedright,
  section/beforeskip = 16pt,
  section/afterskip = 16pt,
  section/fixskip = true,
  subsection/numberformat = \rm,
  subsection/format += \sf\zihao{-4}\raggedright,
  subsection/fixskip = true,
  subsection/beforeskip = 16pt,
  subsection/afterskip = 16pt,
  subsubsection/numberformat = \rm,
  subsubsection/format += \sf\zihao{-4}\raggedright,
  contentsname = {目\quad 录},
}

\usepackage[titles]{tocloft}
\renewcommand{\cftdot}{$\ldotp$}
\renewcommand{\cftdotsep}{0}
\renewcommand{\cftchapleader}{\cftdotfill{\cftchapdotsep}}
\renewcommand{\cftchapdotsep}{\cftdotsep}

\usepackage[lotdepth=2,lofdepth=2]{subfig}

\usepackage{fancyhdr}
\pagestyle{fancy}
\fancyhf{}
\renewcommand{\headrule}{\hrule height1pt width\headwidth \vspace{3.0pt}\hrule width\headwidth}
% \fancyhead[EC]{\kaishu \zihao{5}中国矿业大学(北京)硕士学位论文}
\fancyhead[EC]{\kaishu \zihao{5}https://iotctech.github.io/deeplearning}
\fancyhead[OC]{\kaishu \zihao{5}\leftmark}
\fancyfoot[C]{\thepage}

\fancypagestyle{plain}{ \fancyhf{}
% \fancyhead[EC]{\kaishu\zihao{5} 中国矿业大学(北京)硕士学位论文}
\fancyhead[EC]{\kaishu\zihao{5} https://iotctech.github.io/deeplearning}
\fancyhead[OC]{\kaishu\zihao{5} \leftmark}
\fancyfoot[C]{\thepage}}

\usepackage{array}
\usepackage{multirow}
\usepackage[table]{xcolor}
\usepackage{wrapfig}
\usepackage{float}
\usepackage{colortbl}
\usepackage{pdflscape}
\usepackage{tabu}
\usepackage{threeparttable}
\usepackage{threeparttablex}
\usepackage[normalem]{ulem}
\usepackage{makecell}

\frontmatter

\usepackage[super,square,sort]{natbib}
\bibliographystyle{GBT7714-2005}


\title{深度学习文档集}
\providecommand{\subtitle}[1]{}
\subtitle{Deep Learning Book}
\author{贝塔}
\date{2020-03-22}

\begin{document}
% \maketitle

%\input{latex/00a-cover.tex}  % 封面

%\input{latex/00b-titlepage.tex}  % 标题

%\input{latex/00c-declaration.tex}  % 声明

%\input{latex/00d-abstract.tex}  % 摘要

{
\setcounter{tocdepth}{2}
\tableofcontents
}

\hypertarget{ux5199ux5728ux524dux9762}{%
\chapter{写在前面}\label{ux5199ux5728ux524dux9762}}

\emph{Life, thin and light-off time and time again}

\emph{Frivolous tireless}

生命,一次又一次轻薄过

轻狂不知疲倦

\hypertarget{Alexnet}{%
\chapter{ImageNet Classification with Deep Convolutional Neural Networks}\label{Alexnet}}

\hypertarget{abstract}{%
\section{Abstract}\label{abstract}}

We trained a large, deep convolutional neural network to classify the 1.2 million high-resolution images in the ImageNet LSVRC-2010 contest into the 1000 different classes. On the test data, we achieved top-1 and top-5 error rates of 37.5\% and 17.0\% which is considerably better than the previous state-of-the-art. The neural network, which has 60 million parameters and 650,000 neurons, consists of five convolutional layers, some of which are followed by max-pooling layers, and three fully-connected layers with a final 1000-way softmax. To make training faster, we used non-saturating neurons and a very efficient GPU implementation of the convolution operation. To reduce overfitting in the fully-connected layers we employed a recently-developed regularization method called ``dropout'' that proved to be very effective. We also entered a variant of this model in the ILSVRC-2012 competition and achieved a winning top-5 test error rate of 15.3\%, compared to 26.2\% achieved by the second-best entry

\hypertarget{introduction}{%
\section{Introduction}\label{introduction}}

Current approaches to object recognition make essential use of machine learning methods. To improve their performance, we can collect larger datasets, learn more powerful models, and use better techniques for preventing overfitting. Until recently, datasets of labeled images were relatively small --- on the order of tens of thousands of images (e.g., NORB {[}16{]}, Caltech-101/256 {[}8, 9{]}, and CIFAR-10/100 {[}12{]}). Simple recognition tasks can be solved quite well with datasets of this size, especially if they are augmented with label-preserving transformations. For example, the currentbest error rate on the MNIST digit-recognition task (\textless0.3\%) approaches human performance {[}4{]}. But objects in realistic settings exhibit considerable variability, so to learn to recognize them it is necessary to use much larger training sets. And indeed, the shortcomings of small image datasets have been widely recognized (e.g., Pinto et al.~{[}21{]}), but it has only recently become possible to collect labeled datasets with millions of images. The new larger datasets include LabelMe {[}23{]}, which consists of hundreds of thousands of fully-segmented images, and ImageNet {[}6{]}, which consists of over 15 million labeled high-resolution images in over 22,000 categories.

To learn about thousands of objects from millions of images, we need a model with a large learning capacity. However, the immense complexity of the object recognition task means that this problem cannot be specified even by a dataset as large as ImageNet, so our model should also have lots of prior knowledge to compensate for all the data we don't have. Convolutional neural networks (CNNs) constitute one such class of models {[}16, 11, 13, 18, 15, 22, 26{]}. Their capacity can be controlled by varying their depth and breadth, and they also make strong and mostly correct assumptions about the nature of images (namely, stationarity of statistics and locality of pixel dependencies). Thus, compared to standard feedforward neural networks with similarly-sized layers, CNNs have much fewer connections and parameters and so they are easier to train, while their theoretically-best performance is likely to be only slightly worse.

Despite the attractive qualities of CNNs, and despite the relative efficiency of their local architecture, they have still been prohibitively expensive to apply in large scale to high-resolution images. Luckily, current GPUs, paired with a highly-optimized implementation of 2D convolution, are powerful enough to facilitate the training of interestingly-large CNNs, and recent datasets such as ImageNet contain enough labeled examples to train such models without severe overfitting.

The specific contributions of this paper are as follows: we trained one of the largest convolutional neural networks to date on the subsets of ImageNet used in the ILSVRC-2010 and ILSVRC-2012 competitions {[}2{]} and achieved by far the best results ever reported on these datasets. We wrote a highly-optimized GPU implementation of 2D convolution and all the other operations inherent in training convolutional neural networks, which we make available publicly. Our network contains a number of new and unusual features which improve its performance and reduce its training time, which are detailed in Section 3. The size of our network made overfitting a significant problem, even with 1.2 million labeled training examples, so we used several effective techniques for preventing overfitting, which are described in Section 4. Our final network contains five convolutional and three fully-connected layers, and this depth seems to be important: we found that removing any convolutional layer (each of which contains no more than 1\% of the model's parameters) resulted in inferior performance.

In the end, the network's size is limited mainly by the amount of memory available on current GPUs and by the amount of training time that we are willing to tolerate. Our network takes between five and six days to train on two GTX 580 3GB GPUs. All of our experiments suggest that our results can be improved simply by waiting for faster GPUs and bigger datasets to become available.

\hypertarget{the-dataset}{%
\section{The Dataset}\label{the-dataset}}

ImageNet is a dataset of over 15 million labeled high-resolution images belonging to roughly 22,000 categories. The images were collected from the web and labeled by human labelers using Amazon's Mechanical Turk crowd-sourcing tool. Starting in 2010, as part of the Pascal Visual Object Challenge, an annual competition called the ImageNet Large-Scale Visual Recognition Challenge (ILSVRC) has been held. ILSVRC uses a subset of ImageNet with roughly 1000 images in each of 1000 categories. In all, there are roughly 1.2 million training images, 50,000 validation images, and 150,000 testing images.

ILSVRC-2010 is the only version of ILSVRC for which the test set labels are available, so this is the version on which we performed most of our experiments. Since we also entered our model in the ILSVRC-2012 competition, in Section 6 we report our results on this version of the dataset as well, for which test set labels are unavailable. On ImageNet, it is customary to report two error rates: top-1 and top-5, where the top-5 error rate is the fraction of test images for which the correct label is not among the five labels considered most probable by the model.

ImageNet consists of variable-resolution images, while our system requires a constant input dimensionality. Therefore, we down-sampled the images to a fixed resolution of 256 × 256. Given a rectangular image, we first rescaled the image such that the shorter side was of length 256, and then cropped out the central 256×256 patch from the resulting image. We did not pre-process the images in any other way, except for subtracting the mean activity over the training set from each pixel. So we trained our network on the (centered) raw RGB values of the pixels.

\hypertarget{the-architecture}{%
\section{The Architecture}\label{the-architecture}}

The architecture of our network is summarized in Figure 2. It contains eight learned layers --- five convolutional and three fully-connected. Below, we describe some of the novel or unusual features of our network's architecture. Sections 3.1-3.4 are sorted according to our estimation of their importance, with the most important first.

\hypertarget{relu-nonlinearity}{%
\subsection{ReLU Nonlinearity}\label{relu-nonlinearity}}

tandard way to model a neuron's output \(f\) as a function of its input \(x\) is with \(f(x) = tanh(x)\) or \(f(x) = (1 + e−x )−1\) . In terms of training time with gradient descent, these saturating nonlinearities are much slower than the non-saturating nonlinearity \(f(x) = max(0, x)\). Following Nair and Hinton {[}20{]}, we refer to neurons with this nonlinearity as Rectified Linear Units (ReLUs). Deep convolutional neural networks with ReLUs train several times faster than their equivalents with tanh units. This is demonstrated in Figure 1, which shows the number of iterations required to reach 25\% training error on the CIFAR-10 dataset for a particular four-layer convolutional network. This plot shows that we would not have been able to experiment with such large neural networks for this work if we had used traditional saturating neuron models.

We are not the first to consider alternatives to traditional neuron models in CNNs. For example, Jarrett et al.~{[}11{]} claim that the nonlinearity \(f(x) = |tanh(x)|\) works particularly well with their type of contrast normalization followed by local average pooling on the Caltech-101 dataset. However, on this dataset the primary concern is preventing overfitting, so the effect they are observing is different from the accelerated ability to fit the training set which we report when using ReLUs. Faster learning has a great influence on the performance of large models trained on large datasets.

\begin{center}\includegraphics[width=0.7\linewidth]{img/01-01} \end{center}

Figure 1: A four-layer convolutional neural network with ReLUs \textbf{(solid line)} reaches a 25\% training error rate on CIFAR-10 six times faster than an equivalent network with tanh neurons \textbf{(dashed line)}. The learning rates for each network were chosen independently to make training as fast as possible. No regularization of any kind was employed. The magnitude of the effect demonstrated here varies with network architecture, but networks with ReLUs consistently learn several times faster than equivalents with saturating neurons.

\hypertarget{training-on-multiple-gpus}{%
\subsection{Training on Multiple GPUs}\label{training-on-multiple-gpus}}

A single GTX 580 GPU has only 3GB of memory, which limits the maximum size of the networks that can be trained on it. It turns out that 1.2 million training examples are enough to train networks which are too big to fit on one GPU. Therefore we spread the net across two GPUs. Current GPUs are particularly well-suited to cross-GPU parallelization, as they are able to read from and write to one another's memory directly, without going through host machine memory. The parallelization scheme that we employ essentially puts half of the kernels (or neurons) on each GPU, with one additional trick: the GPUs communicate only in certain layers. This means that, for example, the kernels of layer 3 take input from all kernel maps in layer 2. However, kernels in layer 4 take input only from those kernel maps in layer 3 which reside on the same GPU. Choosing the pattern of connectivity is a problem for cross-validation, but this allows us to precisely tune the amount of communication until it is an acceptable fraction of the amount of computation.

The resultant architecture is somewhat similar to that of the ``columnar'' CNN employed by Cire¸san et al.~{[}5{]}, except that our columns are not independent (see Figure 2). This scheme reduces our top-1 and top-5 error rates by 1.7\% and 1.2\%, respectively, as compared with a net with half as many kernels in each convolutional layer trained on one GPU. The two-GPU net takes slightly less time to train than the one-GPU net2.

\hypertarget{local-response-normalization}{%
\subsection{Local Response Normalization}\label{local-response-normalization}}

ReLUs have the desirable property that they do not require input normalization to prevent them from saturating. If at least some training examples produce a positive input to a ReLU, learning will happen in that neuron. However, we still find that the following local normalization scheme aids generalization. Denoting by \(a^i_{x,y}\) the activity of a neuron computed by applying kernel i at position \((x, y)\) and then applying the ReLU nonlinearity, the response-normalized activity \(b^i_{x,y}\) is given by the expression


\backmatter
% \printindex

\end{document}
