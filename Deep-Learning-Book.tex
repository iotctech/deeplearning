\documentclass[]{book}
\usepackage{lmodern}
\usepackage{amssymb,amsmath}
\usepackage{ifxetex,ifluatex}
\usepackage{fixltx2e} % provides \textsubscript
\ifnum 0\ifxetex 1\fi\ifluatex 1\fi=0 % if pdftex
  \usepackage[T1]{fontenc}
  \usepackage[utf8]{inputenc}
\else % if luatex or xelatex
  \ifxetex
    \usepackage{mathspec}
  \else
    \usepackage{fontspec}
  \fi
  \defaultfontfeatures{Ligatures=TeX,Scale=MatchLowercase}
\fi
% use upquote if available, for straight quotes in verbatim environments
\IfFileExists{upquote.sty}{\usepackage{upquote}}{}
% use microtype if available
\IfFileExists{microtype.sty}{%
\usepackage{microtype}
\UseMicrotypeSet[protrusion]{basicmath} % disable protrusion for tt fonts
}{}
\usepackage[unicode=true]{hyperref}
\hypersetup{
            pdftitle={深度学习文档集},
            pdfauthor={贝塔},
            pdfborder={0 0 0},
            breaklinks=true}
\urlstyle{same}  % don't use monospace font for urls
\usepackage{natbib}
\bibliographystyle{apalike}
\usepackage{longtable,booktabs}
\usepackage{graphicx,grffile}
\makeatletter
\def\maxwidth{\ifdim\Gin@nat@width>\linewidth\linewidth\else\Gin@nat@width\fi}
\def\maxheight{\ifdim\Gin@nat@height>\textheight\textheight\else\Gin@nat@height\fi}
\makeatother
% Scale images if necessary, so that they will not overflow the page
% margins by default, and it is still possible to overwrite the defaults
% using explicit options in \includegraphics[width, height, ...]{}
\setkeys{Gin}{width=\maxwidth,height=\maxheight,keepaspectratio}
\IfFileExists{parskip.sty}{%
\usepackage{parskip}
}{% else
\setlength{\parindent}{0pt}
\setlength{\parskip}{6pt plus 2pt minus 1pt}
}
\setlength{\emergencystretch}{3em}  % prevent overfull lines
\providecommand{\tightlist}{%
  \setlength{\itemsep}{0pt}\setlength{\parskip}{0pt}}
\setcounter{secnumdepth}{5}
% Redefines (sub)paragraphs to behave more like sections
\ifx\paragraph\undefined\else
\let\oldparagraph\paragraph
\renewcommand{\paragraph}[1]{\oldparagraph{#1}\mbox{}}
\fi
\ifx\subparagraph\undefined\else
\let\oldsubparagraph\subparagraph
\renewcommand{\subparagraph}[1]{\oldsubparagraph{#1}\mbox{}}
\fi
\usepackage{booktabs}
\usepackage{amsthm}
\makeatletter
\def\thm@space@setup{%
  \thm@preskip=8pt plus 2pt minus 4pt
  \thm@postskip=\thm@preskip
}
\makeatother

\title{深度学习文档集}
\author{贝塔}
\date{2020-03-22}

\begin{document}
\maketitle

{
\setcounter{tocdepth}{1}
\tableofcontents
}
\chapter{写在前面}\label{ux5199ux5728ux524dux9762}

\emph{Life, thin and light-off time and time again}

\emph{Frivolous tireless}

生命,一次又一次轻薄过

轻狂不知疲倦

\chapter{Introduction}\label{intro}

\section{1.1
标量、向量、矩阵、张量之间的联系}\label{ux6807ux91cfux5411ux91cfux77e9ux9635ux5f20ux91cfux4e4bux95f4ux7684ux8054ux7cfb}

\textbf{标量(scalar)}\\
一个标量表示一个单独的数,它不同于线性代数中研究的其他大部分对象(通常是多个数的数组)。我们用斜体表示标量。标量通常被赋予小写的变量名称。

\textbf{向量(vector)}\\
​一个向量表示一组有序排列的数。通过次序中的索引,我们可以确定每个单独的数。通常我们赋予向量粗体的小写变量名称,比如xx。向量中的元素可以通过带脚标的斜体表示。向量\(X\)的第一个元素是\(X_1\),第二个元素是\(X_2\),以此类推。我们也会注明存储在向量中的元素的类型(实数、虚数等)。

\textbf{矩阵(matrix)}\\
​矩阵是具有相同特征和纬度的对象的集合,表现为一张二维数据表。其意义是一个对象表示为矩阵中的一行,一个特征表示为矩阵中的一列,每个特征都有数值型的取值。通常会赋予矩阵粗体的大写变量名称,比如\(A\)。

\textbf{张量(tensor)}\\
​在某些情况下,我们会讨论坐标超过两维的数组。一般地,一个数组中的元素分布在若干维坐标的规则网格中,我们将其称之为张量。使用
\(A\)
来表示张量``A''。张量\(A\)中坐标为\((i,j,k)\)的元素记作\(A_{(i,j,k)}\)。

\textbf{四者之间关系}

\begin{quote}
标量是0阶张量,向量是一阶张量。举例:\\
​标量就是知道棍子的长度,但是你不会知道棍子指向哪儿。\\
​向量就是不但知道棍子的长度,还知道棍子指向前面还是后面。\\
​张量就是不但知道棍子的长度,也知道棍子指向前面还是后面,还能知道这棍子又向上/下和左/右偏转了多少。
\end{quote}

\section{1.2
张量与矩阵的区别?}\label{ux5f20ux91cfux4e0eux77e9ux9635ux7684ux533aux522b}

\begin{itemize}
\tightlist
\item
  从代数角度讲,
  矩阵它是向量的推广。向量可以看成一维的``表格''(即分量按照顺序排成一排),
  矩阵是二维的``表格''(分量按照纵横位置排列),
  那么\(n\)阶张量就是所谓的\(n\)维的``表格''。
  张量的严格定义是利用线性映射来描述。
\item
  从几何角度讲,
  矩阵是一个真正的几何量,也就是说,它是一个不随参照系的坐标变换而变化的东西。向量也具有这种特性。
\item
  张量可以用3×3矩阵形式来表达。
\item
  表示标量的数和表示向量的三维数组也可分别看作1×1,1×3的矩阵。
\end{itemize}

\section{1.3
矩阵和向量相乘结果}\label{ux77e9ux9635ux548cux5411ux91cfux76f8ux4e58ux7ed3ux679c}

​
一个\(m\)行\(n\)列的矩阵和\(n\)行向量相乘,最后得到就是一个\(m\)行的向量。运算法则就是矩阵中的每一行数据看成一个行向量与该向量作点乘。

\section{1.4
向量和矩阵的范数归纳}\label{ux5411ux91cfux548cux77e9ux9635ux7684ux8303ux6570ux5f52ux7eb3}

\textbf{向量的范数}\\
​
定义一个向量为:\(\vec{a}=[-5, 6, 8, -10]\)。任意一组向量设为\(\vec{x}=(x_1,x_2,...,x_N)\)。其不同范数求解如下:

\begin{itemize}
\tightlist
\item
  向量的1范数:向量的各个元素的绝对值之和,上述向量\(\vec{a}\)的1范数结果就是:29。
\end{itemize}

\[
\Vert\vec{x}\Vert_1=\sum_{i=1}^N\vert{x_i}\vert
\]

\begin{itemize}
\tightlist
\item
  向量的2范数:向量的每个元素的平方和再开平方根,上述\(\vec{a}\)的2范数结果就是:15。
\end{itemize}

\[
\Vert\vec{x}\Vert_2=\sqrt{\sum_{i=1}^N{\vert{x_i}\vert}^2}
\]

\begin{itemize}
\tightlist
\item
  向量的负无穷范数:向量的所有元素的绝对值中最小的:上述向量\(\vec{a}\)的负无穷范数结果就是:5。
\end{itemize}

\[
\Vert\vec{x}\Vert_{-\infty}=\min{|{x_i}|}
\]

\begin{itemize}
\tightlist
\item
  向量的正无穷范数:向量的所有元素的绝对值中最大的:上述向量\(\vec{a}\)的正无穷范数结果就是:10。
\end{itemize}

\[
\Vert\vec{x}\Vert_{+\infty}=\max{|{x_i}|}
\]

\begin{itemize}
\tightlist
\item
  向量的p范数:
\end{itemize}

\[
L_p=\Vert\vec{x}\Vert_p=\sqrt[p]{\sum_{i=1}^{N}|{x_i}|^p}
\]

\textbf{矩阵的范数}

定义一个矩阵\(A=[-1, 2, -3; 4, -6, 6]\)。
任意矩阵定义为:\(A_{m\times n}\),其元素为 \(a_{ij}\)。

矩阵的范数定义为

\[
\Vert{A}\Vert_p :=\sup_{x\neq 0}\frac{\Vert{Ax}\Vert_p}{\Vert{x}\Vert_p}
\]

当向量取不同范数时, 相应得到了不同的矩阵范数。

\begin{itemize}
\tightlist
\item
  \textbf{矩阵的1范数(列范数)}:矩阵的每一列上的元素绝对值先求和,再从中取个最大的,(列和最大),上述矩阵\(A\)的1范数先得到\([5,8,9]\),再取最大的最终结果就是:9。
\end{itemize}

\[
\Vert A\Vert_1=\max_{1\le j\le}\sum_{i=1}^m|{a_{ij}}|
\]

\begin{itemize}
\tightlist
\item
  \textbf{矩阵的2范数}:矩阵\(A^TA\)的最大特征值开平方根,上述矩阵\(A\)的2范数得到的最终结果是:10.0623。
\end{itemize}

\[
\Vert A\Vert_2=\sqrt{\lambda_{max}(A^T A)}
\]

其中, \(\lambda_{max}(A^T A)\) 为 \(A^T A\) 的特征值绝对值的最大值。 -
\textbf{矩阵的无穷范数(行范数)}:矩阵的每一行上的元素绝对值先求和,再从中取个最大的,(行和最大),上述矩阵\(A\)的1范数先得到\([6;16]\),再取最大的最终结果就是:16。

\[
\Vert A\Vert_{\infty}=\max_{1\le i \le n}\sum_{j=1}^n |{a_{ij}}|
\]

\begin{itemize}
\item
  \textbf{矩阵的核范数}:矩阵的奇异值(将矩阵svd分解)之和,这个范数可以用来低秩表示(因为最小化核范数,相当于最小化矩阵的秩------低秩),上述矩阵A最终结果就是:10.9287。
\item
  \textbf{矩阵的L0范数}:矩阵的非0元素的个数,通常用它来表示稀疏,L0范数越小0元素越多,也就越稀疏,上述矩阵\(A\)最终结果就是:6。
\item
  \textbf{矩阵的L1范数}:矩阵中的每个元素绝对值之和,它是L0范数的最优凸近似,因此它也可以表示稀疏,上述矩阵\(A\)最终结果就是:22。\\
\item
  \textbf{矩阵的F范数}:矩阵的各个元素平方之和再开平方根,它通常也叫做矩阵的L2范数,它的优点在于它是一个凸函数,可以求导求解,易于计算,上述矩阵A最终结果就是:10.0995。
\end{itemize}

\[
\Vert A\Vert_F=\sqrt{(\sum_{i=1}^m\sum_{j=1}^n{| a_{ij}|}^2)}
\]

\begin{itemize}
\tightlist
\item
  \textbf{矩阵的L21范数}:矩阵先以每一列为单位,求每一列的F范数(也可认为是向量的2范数),然后再将得到的结果求L1范数(也可认为是向量的1范数),很容易看出它是介于L1和L2之间的一种范数,上述矩阵\(A\)最终结果就是:17.1559。
\item
  \textbf{矩阵的 p范数}
\end{itemize}

\[
\Vert A\Vert_p=\sqrt[p]{(\sum_{i=1}^m\sum_{j=1}^n{| a_{ij}|}^p)}
\]

\section{1.5
如何判断一个矩阵为正定?}\label{ux5982ux4f55ux5224ux65adux4e00ux4e2aux77e9ux9635ux4e3aux6b63ux5b9a}

\begin{itemize}
\tightlist
\item
  顺序主子式全大于0;\\
\item
  存在可逆矩阵\(C\)使\(C^TC\)等于该矩阵;
\item
  正惯性指数等于\(n\);
\item
  合同于单位矩阵\(E\)(即:规范形为\(E\))
\item
  标准形中主对角元素全为正;
\item
  特征值全为正;
\item
  是某基的度量矩阵。
\end{itemize}

\chapter{Literature}\label{literature}

Here is a review of existing methods.

\chapter{Methods}\label{methods}

We describe our methods in this chapter.

\chapter{Applications}\label{applications}

Some \emph{significant} applications are demonstrated in this chapter.

\section{Example one}\label{example-one}

\section{Example two}\label{example-two}

\chapter{Final Words}\label{final-words}

We have finished a nice book.

\bibliography{book.bib,packages.bib}

\end{document}
